\section{Introduction}

%\nocite{NCDC2012}

One of the most important centers of forest diversity in North America is the Southern Appalachian region. This region has supported continuous forest communities longer than any other area on the continent and hosts many rare, endemic species \cite{NCNHP2012}. Additionally, it harbors many disjunct species populations, all of which make it one of the most important centers of forest diversity on the continent. The southern Appalachians also provide ecosystem services such as carbon storage, watershed and water quality protection, and serve as a timber source \cite{zipper2011restoring}. In order to protect these valuable resources, it is crucial that we thoroughly understand the past climate of this area and how it has influenced the many ecosystems within the region.  A sound understanding of the past relationship between climate and southern Appalachian ecosystems will enable scientists and landowners to better manage the natural resources in the future. 

Global circulation models project an increase in average global surface temperatures of $1.0-3.5^{\circ}$ by the end of this century due to continued increases in greenhouse-gas emissions \cite{pachauri2007climate, kattenberg1996climate}. However the influence of increased radiative forcing on precipitation regimes is not well understood, and this is particularly the case for the southeastern United States (US). The 24 models used to make predictions about climate change in the Intergovernmental Panel on Climate Change Fourth Assessment Report were not in consensus with respect to drought frequency \cite{pachauri2007climate, seager2009drought}.  Uncertainty in climate projections makes it difficult to predict water and power usage. The ability to do so is crucial because the southeastern US has experienced substantial increases in population and energy consumption, over the last decade \cite{seager2009drought, sobolowski2012evaluation}. It is important that the public and planners in the Southeast have access to information regarding climate change projections and mitigation. Through the use of tree-ring based climate reconstructions, scientists may better understand past precipitation regimes at decadal- to centennial time-scales in order to better project future precipitation patterns in a changing climate. 

In order to reduce uncertainty in climate model projections and to extend meteorological records further back in time, tree-ring data are commonly used as regional proxies, particularly in regions where drought (e.g. the American Southwest, \cite{cook2004long}) or summer temperature (e.g. the European Alps, \cite{buntgen2007growth}) is the limiting tree growth factor. However, tree-ring data have also successfully been used for climate reconstructions in the eastern US \cite{leblanc1993temporal, stahle1993, cook1999drought}. Traditionally it has been understood that trees in a closed-canopy forest are not limited by climate to the same extent as trees growing on the forest border \cite{fritts1976tree}. Within a dense forest, stand dynamics play an important role in shaping the forest structure through their influence on radial tree growth and tree survival. As these interactions between individuals increase in strength, the climatic influence on tree growth becomes less dominant. 

Trees growing in temperate regions characterized by high humidity such as those in the Southeast US are typically thought to be less sensitive to climate than trees in semiarid regions \cite{phipps1982comments}. This belief supports the idea that the degree to which an environmental factor is limiting affects the amount of variability in that factor that is seen in tree-ring time series. Although water access may not be limiting in southeastern US sites, a large sample size may compensate to help identify the common climate signals despite site and individual variability. In regions that are subject to site heterogeneity, where significant climatic variance cannot be identified for a standard sample size, principal component analysis can be an effective means to overcome the lack of strength of climate signal \cite{peters1981principal, anchukaitis2006forward, jacoby1989reconstructed}. Through the application of principal component analysis (PCA)\label{sym:PCA}, tree-ring data collected from a network of regional sites can be combined to reduce-site level noise through the identification of a common climate signal across sites.

Despite the challenges of finding a climate signal in tree-ring time series in southeastern US forests, numerous studies have identified climate-growth correlations \cite{pan1997dendroclimatological, speer2009climate, rubino2000dendroclimatological}. For example, Pan et al. \citet{pan1997dendroclimatological} showed that after tree-ring standardization, both annual ring-width and basal area increments of four deciduous species in Virginia were positively correlated with precipitation from both the prior summer, autumn, and current summer. They also report negative correlations with air temperature of the current growing season. Speer et al. \citet{speer2009climate} found similar correlations between precipitation and temperature and annual tree growth for oak chronologies from closed canopy forest in the Southern Appalachian Mountains. 

In this study, we determined the presence of a significant relationship between chestnut oak growth-series in the eastern US and early summer precipitation and ascertained the viability of a climatic reconstruction based on the chestnut oak growth series as proxy data. The annual growth proxy data was subsequently used to reconstruct early summer precipitation using Bayesian methods. Finally, I evaluated the reliability of the reconstruction by comparing it to other verified regional reconstructions.  

%%
%% methods
%%
\section{Materials and Methods}
\label{sec:meth}

\subsection{Tree ring data}

The study site was an Upland Oak-Pine forest located on the North facing slope of Brush Mountain in South-West Virginia  ($37^{\circ} \ 22.2$' N, $80^{\circ}\ 14.8$' W), with a site elevation of 558 m (Figure~\ref{fig:map}). This region is classified as either humid continental or mountain temperate, and characterized by warm, humid summers and winters that are predominantly cool with intermittent warm spells. The mean annual precipitation 1901-2010 at Blacksburg weather station was 1073 mm and the mean annual temperature was $10.9^{\circ} C$. 

The study site supported older chestnut oak trees amongst a canopy of many species, including scarlet oak (\textit{Quercus coccinea}), northern red oak (\textit{Quercus rubra}), red maple (\textit{Acer rubrum}), Virginia pine (\textit{Pinus virginiana}), pitch pine (\textit{Pinus pungens}), and eastern white pine (\textit{Pinus strobus}). Site access was adjacent to the Appalachian trail, but the site was selected to minimize human interference. The steepness of this slope suggested that climate may be a limiting growth factor, although the closed canopy and stand density suggested that stand dynamics may also play a significant role in shaping the forest structure.

Two cores were collected from each of the 56 chestnut oak trees sampled. Samples were dried, mounted, and sanded according to standard guidelines \cite{stokes1996introduction}. Crossdating was performed using reflected light microscopy and the list method, which facilitates the identification of marker years that signify relatively favorable or unfavorable growth years in a stand \cite{yamaguchi1991simple}. All samples were measured using a LINTAB measurement stage with 0.01mm precision, and visual crossdating was checked using COFECHA \cite{holmes1983computer}. COFECHA uses segmented time series correlation techniques that make use of common variability present in samples from a given site to identify potential crossdating errors \cite{grissino2001research}. COFECHA also computes inter-series correlation, which is a measure of stand-level signal, and mean sensitivity, which measures the year-to-year variability in a time series. Based on inter-series correlation coefficients, a total of 76 tree-ring series from 53 trees contained enough common growth signal to be used for further analysis.

Non-climatic age-dependent and stand-dynamics related trends were removed from the individual tree-ring series using smoothing splines with a 50 \% cutoff at 50 years (ARSTAN software, \cite{cook1997calculating}). This method allowed us the flexibility to remove the episodic-like interaction effects from the time series, while retaining the high-frequency climatic variability. Note that as with any filtering technique, inevitably some portion of the climatic signal will be lost through the removal of these non-climatic trends \cite{cook1981smoothing}.  I  here assume that the loss of climatic signal was negligible, and comparison of the detrended time series with climatic data ultimately determined if the strength of the remaining signal was sufficient to perform further analyses. Furthermore, serial correlation is common in tree-ring time series, typically due to the availability of stored water or photosynthates. This autocorrelation effectively reduces the number of independent observations, and therefore must be taken into account through either reduction of the effective sample size to ensure that observation independence, or through autoregressive and/or moving average (ARMA)\label{sym:ARMA} modeling \cite{monserud1986time, cook1987decomposition}. All series were checked for autocorrelation to determine if prewhitening via ARMA modeling was necessary, and applied when deemed necessary. The Brush Mountain site chronology was then developed based on the individually detrended tree-ring width time series, and will hereafter be referred to as BM. \label{sym:BM}

I also computed the expressed population signal (EPS)\label{sym:EPS} to measure the common variability in the chronology at an annual resolution. EPS depends on both signal coherence and annual sample-depth. When EPS values which fall below a predetermined cutoff (0.85), the chronology is no longer dominated by a coherent signal, and is therefore deemed less than ideal for climatic reconstructions. 


\subsection{Principal component analysis}

It is often the case that data from a single closed-canopy site does not show a strong relationship with climate. In this case, data from additional sites may provide some insight into the regional climate signal through the use of principal component analysis (PCA). PCA can help identify common patterns in climate-modulated tree growth between sites and reduce the dimensionality of the data. A total of 8 \textit{Quercus prinus} chronologies from the eastern US obtained from the International Tree-Ring Database (ITRDB) were considered for inclusion in a PCA analysis. Chronology reliability for each of the 8 chronologies was assessed based on the mean sensitivity, inter-series correlation, the EPS, and the first-order autocorrelation.
For each considered site, raw ring-width time series were detrended using a smoothing spline with 50\% cutoff at 50 years, and subsequently used to build chronologies for each of the respective sites. These chronologies were then considered for use in our PCA with the goal of developing a stronger climatic signal. Only chronologies which extended back to at least 1845 and which were significantly correlated with precipitation anomalies were retained for further analysis. A set of 4 nearby tree-ring chronologies (3 \textit{Quercus prinus} and 1 \textit{Quercus alba}) met these conditions (Table~\ref{table:chronStats},Fig.~\ref{fig:stackedChrons}), and were combined with the BM chronology in a nested singular value decomposition PCA \cite{wold1987principal}. The first PCA (5 contributing chronologies) was performed on the 1845-1981 time interval, and the second PCA (4 contributing chronologies) on the 1750-1981 interval. The PCA components with eigenvalues larger than one were retained for further analysis and the components explaining the largest amount of common variance in the tree-ring chronologies were included in a climate correlation analysis.

\subsection{Climate data}

Monthly precipitation, mean temperature, as well as mean Palmer Drought Severity Index\label{sym:PDSI} (PDSI, Palmer 1965) were computed from daily measurements at the Blacksburg climate station ($37^{\circ} \ 12$' N, $80^{\circ}\ 24$' W; elevation 634 m; 1901-2006). PDSI is an index of drought severity that is based on a simplified water balance equation \cite{wells2004self}. This method requires that for each month of the year, soil moisture and water potential values are computed and then used to copmute an excess or shortage of precipitation when compared to the precipitation the is deemed climatically appropriate for existing conditions. When this result is multiplied with something called a climatic characterisitc which allows the measure to be standardized across space, the result is the moisture index. 

Precipitation, temperature and PDSI were all used in a correlation function analysis with the PCA time series. Pearson's correlation coefficients were calculated for all months starting in April of the year previous to the growing season through current December, as well as for the seasons (Apr-June, July-Sep, Oct-Dec, Jan-Mar) and annual means.

The Blacksburg station monthly/seasonal climate variable with the strongest correlation with the BM chronology was then used as guidance for a spatial correlation analysis using a gridded ($0.5^{\circ} \times 0.5^{\circ}$) monthly climate data set for the period 1901-2006 \cite{mitchell2005improved}. The grid point with the strongest correlation coefficient was then used as a target reconstruction.

%Temperature, precipation, and PDSI compute from instrumental measurements taken from the Blacksburg climate station  were compared to the BM chronology to indentify any significant correlations (Pearson's correlation coefficients). This information was used to guide the comparison between the BM chronology and high-resolution gridded ($0.5^{\circ} \times 0.5^{\circ}$) monthly CRU climate data, which allowed us to compare grid points of locations with higher elevations. Regions of significant correlation were examined using higher resolution CRU data ($2.5^{\circ} \times 2.5^{\circ}$). Cross correlations were computed between the BM time series and each of the monthly series for temperature, precipitation, and PDSI beginning with April of the previous growing season through current December for the $1901 - 1981$ period.

\subsection{Reconstruction methods}

To perform the reconstruction of computed climate variable anomalies,  I  use Bayesian linear regression with the selected principal components as proxies. I assume that the precipitation anomalies (y) satisfy $y_t \sim Normal( \mu_t, \sigma^2)$, where $\mu_t = \beta_0 + \beta_1 x_t$ where $x_t$ is the first principal component value at year $t$. In a Bayesian regression formulation  I  make the assumption that the true parameter values $\beta_0$, $\beta_1$, and $\sigma^2$ are distributed according to a probability distribution function (PDF), and that these distributions express the degree of belief about where the true values lies. In a Bayesian framework, the PDFs are approximated by the posterior distribution, which is proportional to the likelihood multiplied by a prior. Posterior distributions can either be sampled directly if a closed-form solution exists, or can be indirectly sampled using a Markov Chain Monte Carlo (MCMC)\label{sym:MCMC} algorithm. Due to the absence of prior information, parameters are assigned uninformative priors which take the form $\beta_i \sim \text{Normal}(\vec{0},1000)$ and $\sigma^2 \sim \text{Uniform}(0,100)$. These uninformative priors indicate that I assign approximately equal weight to all possible parameters values because there was no reason to assume that any specific value is more likely than another. Model parameter distributions were determined using an MCMC algorithm with a Metropolis step method, and was run for 100,000 iterations with a burn-in of 50,000 which I found was more than sufficient to ensure convergence. For the sense of practicality, parameter estimates were thinned so that only every tenth estimate was saved to memory. The output from the MCMC algorithm generates a chain of parameter values sampled from the posterior distribution, and computing the 0.025, 0.5 and 0.975 quantiles of these chains allows us to define an upper and lower bound for a 95\% credible interval as well as the median for that parameter (which allows us to say that the true parameter has a 0.95 probability of falling within that credible interval). For each set of sampled parameters, I generate predicted precipitation values for the years 1745 though 1981 according our model using our growth proxy principal component values ($x_t$), and similarly define a 95\% predictive interval using quantiles. This method allows us to estimate the uncertainty associated with our predictions based on our model. 

\subsection{Model calibration and verification}

To assess the accuracy of the modeled precipitation anomalies,  I  split the data into two periods: 1901-1940, and 1941-1981. Both the 1901-1940 and 1941-1981 periods of data were used in turn as the calibration period, to determine if the accuracy of the reconstruction was sufficient to warrant further analysis. Data from the period not used for calibration served as verification data, and for both calibration/verification pairs I computed the mean squared error (MSE)\label{sym:MSE}, reduction of error (RE)\label{sym:RE} \cite{fritts1976tree}, coefficient of efficiency (CE)\label{sym:CE} \cite{cook1994spatial}, and the squared correlation ($r^2$)\label{sym:r2} (See the National Research Council report Surface Temperature Reconstructions for the Last 2,000 Years \cite{national2006surface} for further details on assessing reconstruction skill). Lastly, I computed the sign test or Gleichl\"{a}ufigkeit (GLK)\label{sym:GLK} score which measures the similarity of the relative annual change in value between two time series \cite{speer2010fundamentals, schweingruber1988tree}.  

\subsection{Reconstruction assessment}

To identify any dominant cyclical behavior in the reconstruction,  I  use a periodogram to calculate the significance of different frequencies in our time series. In the periodogram, peaks in the estimated spectrum are tested to determine if they are different from the underlying white-noise spectrum. Spectrum values are averaged with 2 frequencies per bin to simplify interpretation.  

The precipitation reconstruction was compared to other regional precipitation and drought reconstructions as external validation. For the southeastern US  I  identified a total of six published reconstructions that were used for comparison (Table~\ref{table:reconDeets}). Out of these six, two were drought reconstructions. The first was obtained from the North American Drought Atlas \cite{cook1999drought} which is a gridded reconstruction of PDSI values for June through August (NADA)\label{sym:NADA}, while the second is a July PDSI reconstruction (JT)\label{sym:JT} for Virginia and North Carolinian coastal regions developed by Stahle and Cleaveland \cite{stahle1998lost}. The remaining four reconstructions identified for comparison were precipitation reconstructions. The first set were developed by Stahle and Cleaveland for the North Carolina (NC)\label{sym:NC}, South Carolina\label{sym:SC} (SC), and Georgia\label{sym:GA} (GA) regions for the months of April though June for NC and March through June for SC and GA \cite{stahle1992reconstruction}. The second precipitation reconstruction for early summer anomalies (MP)\label{sym:MP} was developed by Druckenbrod \cite{druckenbrod2003late}. 

%%
%% results
%%

\section{Results}

The BM chronology covered the years 1764-2010, had an interseries correlation of 0.556 and a mean sensitivity of 0.208 (Table~\ref{table:chronStats}). The EPS was greater than the 0.85 cutoff for the years 1845-1981. The highest correlation between the BM chronology and the meteorological weather data was with average May-June precipitation (mjPR\label{sym:mjPR}) or average June-July PDSI (jjPDSI\label{sym:jjPDSI}). To increase the signal to noise ratio in our tree-ring record and better identify the regional precipitation effects, I included four oak chronologies from nearby locations in our analysis. These four chronologies met the requirements for inclusion in the proceeding analysis: they were significantly correlated with mjPR\label{sym:mjPR} or jjPDSI\label{sym:jjPDSI} and covered at least the same time period as the BM chronology (1845 - 1981) as shown in table~\ref{table:chronStats}). The locations and time series of the suitable chronologies, hereafer referred to by abbreviations of their locations as LH, WD, CC, and OC,\label{sym:LH}\label{sym:WD}\label{sym:CC}\label{sym:OC} are shown in figures~\ref{fig:map} and \ref{fig:stackedChrons}. A nested PCA was performed using these identified chronologies in addition to BM. 

The first PCA was performed on all five chronologies for the overlapping time period 1845-1981 determined by the BM EPS values, resulting in a first principal component which explained 57.0\% of the common variance, and a second component which explained 15.2\%. The scores plot (Fig.~\ref{fig:scores}) illustrates the relationship between the five chronologies with respect to the first two principal components: all chronologies have a positive score along PC1 while only BM, WD, and LH have a positive score along PC2. The second PCA was performed on the subset of four chronologies which extended back to the year 1750 (LH, WD, CC, OC), and in this case the first principal component explained 48.6\%, while the second component explained 29.1\%. The relationship between these four chronologies is shown in (Fig.~\ref{fig:scores}). All chronologies have a positive score along PC1 and a negative score along PC2. Overlapping portions of the first principal components that resulted from both decompositions were compared via correlation to confirm that both of these were in fact accounting for the same independent axis ($r=0.93$, $p < 0.01$). First principal components were merged at the year 1845 (PCA2: 1750-1844, PCA1: 1845-1981) to form a single proxy record extending from 1750 to 1981. The sample depth for each year for the resulting principal component growth proxy is shown in Figure~\ref{fig:sampleDepth}.


The BM chronology had multiple significant correlations with monthly precipitation and PDSI values, from May of the previous year to December of the current year (Fig.~\ref{fig:climCorr}). I found significant positive correlations between BM and precipitation of previous year June and current year May and June. The strongest correlation was found with average precipitation of the months May and June ($r=0.50$, $p<0.01$). Correlations with PDSI were significantly positive, particularly during the May through August growing season, with the strongest correlation being with average June and July PDSI ($r=0.55$, $p<0.01$). In general, BM correlations with temperature were not significant, except for the previous July, which was negative ($r=-0.19$, $p<0.05$). Both average May-June precipitation (mjPR) and average June-July PDSI (jjPDSI) were considered as candidate climatic targets for reconstruction. The BM chronology was strongly correlated with both mjPR and jjPDSI, but an assessment of the reconstruction verification statistics (results not shown) suggested that the accuracy of a reconstruction based on this proxy data may not be sufficient.  

When comparing the merged PCA time-series with monthly climate variables, I generally find stronger correlations than for BM (Table~\ref{table:moistureCorrs}, Fig.~\ref{fig:climCorr}). This is particularly true for average precipitation for the months May and June ($r=0.61$, $p<0.01$) and average PDSI for the months June and July ($r=0.63$, $p<0.01$).
%The first principal component explained much of the variation, and was more highly correlated with moisture (both with mjPR and jjPDSI).

In the next step, I tested mjPR and jjPDSI as potential reconstruction targets in a split calibration/verification scheme (\cite{fritts1990methods}; Table~\ref{table:reconStats}). Overall calibration and verification $R^2$ statistics (0.56 - 0.64) and GLK values (0.55-0.79) were strong. However, when the earlier period (1901-1941) was used as the calibration period, the RE and CE statistics for mjPR were low but greater than 0; these statistics were higher for jjPDSI for the same period. RE and CE were also higher for mjPR using the later period (1942-1981), whereas values for jjPDSI were negative indicating a poor fit of the reconstruction model. RE and CE are key statistics to determine the skill of a reconstruction, and our decision to reconstruct early summer (May-June) precipitation rather than summer PDSI was based on these values. Our final mjPR reconstruction was calibrated against the entire 1901-1981 interval.

%Reconstruction accuracy statistics showed that although both the early and late calibration periods had RE and CE values greater than 0, the later calibration period (1941-1981) had overall better ability to predict throughout the entire time period for which their was weather data available. These positive statistics suggest that the relationship is indeed both linear and stationary, at least throughout the 1901-1981 period. To increase the length of the calibration period as well as to use a calibration period closer in time to the time frame for which we wished to reconstruct, we used the entire 1901-1981 portion of first principle component time series as proxy data for mjPR in our Bayesian linear model. Despite our encouraging reconstruction statistics and high correlation between the proxy and climate variable, our model fit generate wide credible intervals which showed the uncertainty associated with the reconstruction. 

Posterior parameter distributions for the Bayesian linear regression model were determined via the adaptive Markov chain Monte Carlo (MCMC) algorithm. Parameter means and credible intervals are shown in Table \ref{table:mcmcResults}. The adaptive MCMC algorithm updated proposal distributions accordingly when acceptance rates fell outside ideal range of 0.2-0.5, which ensured that there was good mixing. Each iteration of the algorithm generated a set of parameters from the posterior, and predicted precipitation values were computed for each set from the resulting parameter chains for all years. Predicted precipitation quantiles were used to obtain a 95\% credible interval for average May-June precipitation for each year. Annual predicted precipitation means were also computed. The resulting precipitation reconstruction covers the years from 1750 through 1981, which allows us to extend the instrumental mjPR record back 150 years. Figure~\ref{fig:precipRecon} shows a plot of the reconstruction and the associated uncertainty as described by the 95 \% quantiles for the 1750-1981 period as well as the available averaged May-June precipitation data. Despite our encouraging reconstruction statistics and high correlation between the proxy and climate variable, our model fit generated wide credible intervals which showed the uncertainty associated with the reconstruction. 

The periodogram shows peaks with significant power at approximately 11, 17, and 24 years, as shown in Figure~\ref{fig:spectral}. These peaks indicate that the reconstruction contains strong sinusoidal components with periods corresponding to the observed peaks. 

The mjPR reconstruction correlated significantly positively with four other reconstructions, namely JT (July PDSI reconstruction), NADA (average summer PDSI), NC (early summer precipitation), and MP (early summer precipitation) (Table~\ref{table:reconDeets}). Reconstructions were compared for the overlapping period 1750-1981, except in the case of MP which only covered the period 1784-1966. The strongest correlation was found between our mjPR reconstruction and the NADA drought reconstruction ($r=0.59$, $p < 0.01$), followed by the MP early summer precipitation anomalies reconstruction ($r=0.38$, $p<0.01$), and the NC averaged April-May-June precipitation reconstruction ($r=0.23$, $p<0.01$). Figure~\ref{fig:allRecons} shows the annual and decadal variability in the mjPR and six compared moisture reconstructions for the entire period covered by the longest reconstruction, while Figure~\ref{fig:allReconsZoom} shows the same variability but only for the years which overlap with 1750-1981 time period covered by the mjPR reconstruction. 

%%
%% discussion
%%

\section{Discussion}


In this study,  I  investigated the relationship between climate and annual radial growth of \textit{Quercus prinus} growing at a closed canopy site in the southeastern US. After removing the portion of the signal attributed to stand dynamics and intrinsic age trends,  I  found that the BM chronology was most strongly positively influenced by early summer (May through July) moisture from the year of ring formation. Similar climate-growth relationships have been identified by previous oak studies in the southeastern US \cite{speer2009climate, li2011dendroclimatic} and can be explained by ecophysiological mechanisms. 
%, Estes 1970; Blasing and Duvick 1984; Jacobi and Tainter 1988; Graumlich 1993; Tardif et al. 2006}. 
Radial growth of oak species typically starts in April or May after leaf-out, and even in wetter years is 90\% complete by the end of July \cite{robertson1992factors}. In earlier months of the growing season, carbon is allocated predominantly to radial thickening, while later in the season the focus of this allocation is shifted to carbohydrate storage \cite{zweifel2006intra}. Under severe moisture stress, oak carbon allocation is shifted from shoot to root, thereby increasing the root/shoot ratio \cite{dickson1996oak}. \textit{Quercus prinus} is considered to be more tolerant to drought stress than other oak species and exhibits several morphological adaptations in order to better cope with moisture stress events \cite{dickson1996oak}, but  I  found that its radial growth was strongly influenced by moisture availability. This suggests that in years with inadequate moisture, radial growth is not a priority, and carbon allocation is likely focused on maintenance or root development. The identifiable moisture-response in the detrended BM chronology demonstrates that oaks in a closed-canopy forest can indeed be used to generate paleoclimatic data, although care must be taken when removing the non-climatic portion of the signal. The assumption that the low frequency component of a tree-ring time series is solely attributed to age and stand dynamics trends may not be valid over long-time, and some climatic signal will inevitably be lost during detrending. Without the ability to measure age and stand dynamic effects independently of climatic effects, the effects of filtering via various detrending methods are difficult to evaluate. The development of a biologically motivated trend removal algorithm may improve current practices in dendroclimatology. In addition, care must be taken in closed canopy forests when attempting to use growth series as proxy records as younger stands in the stem-exclusion phase may be dominated by the effects of competition \cite{oliver1980forest}. 

To isolate and strengthen the moisture-growth relationship of the BM chronology  I  performed a nested principal component analysis on regional \textit{Quercus} chronologies that also showed significant correlations with early summer moisture. Five chronologies were included in the principal component analysis which increased the strength of the BM climate signal. The strong relationship of the first principal component (PC1) with early summer moisture is depicted in Figure~\ref{fig:precipCorrMap}. The spatial pattern of this relationship indicates that the tree-growth proxy PC1 is most influenced by moisture in the Great Appalachian Valley. Mountains play an important role in the hydrological cycle for several reasons, one of which being that they are the points of origin of most rivers \cite{beniston1997climatic}. Increases in precipitation in mountainous regions leads to increased stream flow volumes and surface runoff, which in turn increases soil moisture in the Appalachian watershed.

The average May-June precipitation reconstruction (mjPR) showed anomalies consistent with the instrumental precipitation record for 1901-1981, as shown in Figures~\ref{fig:wetdry} and \ref{fig:wetdry_dots}. In particular, the reconstruction correctly identifies the severe nation-wide dust bowl-era drought in the 1930s, the 1954 drought, as well as the dry spell in the 1970s. Other notable years of low early summer precipitation seen in both the instrumental record as well as the reconstruction are 1911, 1914, and 1925.  I  also note the agreement of extreme precipitation in the years 1928, 1942, and 1950, where documented flooding occurred in the southeastern US. All of the late-spring/early-summer anomalies have been observed across the southeastern US in instrumental records, except for the dry spell in the 1970s \cite{edwards1997characteristics}. In years prior to the instrumental record, the mjPR reconstruction identifies several dry periods, as shown in Table~\ref{table:anoms}, most of which have been observed in other moisture reconstructions for the US.  I  also note that as  I  extend the reconstruction further back in time, reliability inevitably decreases as a result of the decrease in sample size.

Our reconstruction shows similar variability when compared to other reconstructions of moisture variability in the southeastern US (Table~\ref{table:reconComps1}). The strongest similarity was found with the NADA PDSI Cook reconstruction. A comparison of the relationship between these two moisture reconstructions identified two time periods in which reconstruction values were not consistent, as shown in figure \ref{fig:reconCompare} where the reconstructions are standardized prior to plotting to highlight similarities and discrepancies. 

The first anomaly occurred in 1774, which our reconstruction identified as an early summer drought. Although this year showed low moisture relative to other years in several other regional reconstructions, the suggestion of drought was not as defined. A closer examination of the chronologies on which the principal component analysis was performed revealed that this drastic departure from the mean MJ precipitation signal was propagated through the PCA by the Craig Creek chronology. Since this year was not identified as a local minimum growth year in other chronologies  I  assume that this drastic reduction in growth at the Craig Creek site is site-specific, and is attributed to local disturbance with a localized effect on growth. 

The second anomaly manifested itself during the 1853 through 1866 period, where the correlation between both the NADA reconstruction and the mjPR reconstruction becomes no longer significant as shown by a 31 year windowed correlation (see figure~\ref{fig:reconRunningCorr}). With the goal of better understanding this anomaly,  I  return to the five chronologies. A plot of a 31 year windowed correlation between each of the regional oak chronologies and the NADA PDSI reconstruction shows that all five chronologies show this same pattern of reduced correlation with the NADA PDSI reconstruction during these years, as shown in figure~\ref{fig:cookRunningPdsiCorr}. These years correspond with the persistent drought near 1860, which also coincides with a La Nina event which occurred from 1855 - 1863. Although La Nina effects are typically seen on the West Coast, these events have effects on weather patterns throughout North America, and have even been shown to effect the Atlantic hurricane season \cite{pielke1999nina}. As opposed to being driver by moisture availability, tree growth during these years was likely driven by the combination of high temperatures and low moisture availability brought on by the large-scale ocean-atmosphere phenomenon. %This shift in moisture limited growth to temperature limited growth suggests that oaks may

%Cole et al., 2002; Cook et al., 2004;
%Herweijer et al., 2006; Hoerling and Kumar, 2003;
%Schubert et al., 2004; Seager et al., 2005]  

Spectral analysis of the precipitation reconstruction identified a dominant 11-year cycle, a periodicity that has been observed in both instrumental and paleo-reconstructed temperature and moisture indices \cite{hancock1979cross, lassen1995variability}. In particular, this cyclic pattern has been identified in June precipitation in the south-eastern US \cite{hancock1979cross}, but was not apparent in western US PDSI tree-ring based reconstructions \cite{cook1997new}. This observed 11-year periodicity is a hallmark characteristic of the solar cycle, which has been shown to be associated with terrestrial climate, identified as one of the contributing factors that determine global temperature \cite{reid2002solar, national1994Solar, lassen1995variability}. Solar periods of high and low activity can be measured by the number of sunspots or the solar cycle length. A larger number of sunspots indicates greater solar activity, and the magnetic fields in these sunspots have the ability to release large amount of stored energy as solar flares or coronal mass ejections, and these changes in released energy in turn affect the realized weather patterns.  Studies have shown that these changes in released energy may also influence hydroclimate \cite{nichols2012hydroclimate, hancock1979cross}. However, despite the presence of strong correlations between terrestrial climate records and solar cycles, physical mechanisms which explain the effects of external solar forcing on global circulation patterns have yet to be fully understood \cite{franks2002assessing}. 
